
% The definitions can be placed anywhere in the document body
% and their order is sorted by <symbol> automatically when
% calling makeindex in the makefile
%
% The \glossary command has the following syntax:
%
% \glossary{entry}
%
% The \nomenclature command has the following syntax:
%
% \nomenclature[<prefix>]{<symbol>}{<description>}
%
% where <prefix> is used for fine tuning the sort order,
% <symbol> is the symbol to be described, and <description> is
% the actual description.

% ----------------------------------------------------------------------
% Roman symbols [r]
\nomenclature[rE]{$\mathbf{E}$}{Electric field vector.}
\nomenclature[rB]{$\mathbf{B}$}{Magnetic field vector.}
\nomenclature[rS]{$\mathbf{S}$}{Poynting vector.}
\nomenclature[rL]{$\mathcal{L}$}{Lagrangian density.}
\nomenclature[rfour]{$\mathcal{F}[\quad]$}{Fourier transform.}
\nomenclature[rIm]{$\Im(\quad)$}{Imaginary part.}
\nomenclature[rRe]{$\Re(\quad)$}{Real part.}

% ----------------------------------------------------------------------
% Greek symbols [g]
\nomenclature[gh]{$\Theta(\quad)$}{Heaviside theta step function.}
\nomenclature[gs]{$\sigma$}{Conductivity.}
\nomenclature[gr]{$\rho$}{Mass density.}
\nomenclature[gy]{$\phi$}{Azimuthal angle.}
\nomenclature[gh]{$\theta$}{Polar angle.}
\nomenclature[gz]{$\omega$}{Frequency.}

% ----------------------------------------------------------------------
% Subscripts [s]
\nomenclature[s]{$x,y,z$}{Cartesian coordinates indices.}


% ----------------------------------------------------------------------
% Supercripts [t]
\nomenclature[t]{$\rm T$}{Transpose.}
\nomenclature[t]{$\dagger$}{Conjugate transpose (Hermitian conjugate).}
\nomenclature[t]{$\ast$}{Complex conjugate; adimensional quantity}
\nomenclature[t]{$k$}{Computational index for time step.}
\nomenclature[t]{$'$}{Derivative.}


